\subsection{Các cơ chế đồng thuận khác}

Ngoài hai cơ chế đồng thuận phổ biến PoW và PoS, có rất nhiều các thuật toán đồng thuận khác. \textit{Bằng chứng bộ nhớ}\footnote{Proof of Capacity - PoC} (PoC) là một cơ chế cho phép chia sẻ không gian bộ nhớ của một nút cho mạng chuỗi khối. Nút nào có càng nhiều không gian bộ nhớ, nút đó càng có nhiều quyền duy trì mạng. \textit{Bằng chứng hoạt động}\footnote{Proof of Activity - PoA} (PoA), được sử dụng trên chuỗi khối \textit{Decred}, là một cơ chế kết hợp giữa PoW và PoS. \textit{Bằng chứng tiêu thụ}\footnote{Proof of Burn - PoB} (PoB) lại yêu cầu các nút gửi lượng nhỏ tiền của chúng tới một địa chỉ ví không thể truy cập. Một cơ chế khác là \textit{Bằng chứng lịch sử}\footnote{Proof of History - PoH} (PoH), được phát triển bởi \textit{Solana Project}, tương tự như cơ chế \textit{Bằng chứng thời gian còn lại}\footnote{Proof of Elapsed Time - PoET} (PoET), mã hoá thông tin thời gian trôi qua để đạt được sự đồng thuận mà không cần tiêu tốn nhiều tài nguyên.