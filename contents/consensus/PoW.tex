\subsection{Bằng chứng công việc - PoW}

\textit{Bằng chứng công việc}\footnote{Proof of Work - PoW} (PoW) là một dạng của bằng chứng mã hoá, trong đó một bên chứng minh cho các bên khác (bên xác thực) rằng họ đã bỏ ra khối lượng tính toán nào đó. Bên xác thực sẽ tuần tự xác minh tính đúng đắn của bằng chứng này một cách dễ dàng.\\

PoW được đưa ra lần đầu bởi \textit{Cynthia Dwork} và \textit{Moni Naor} vào năm 1993 như một cách để xác định các cuộc \textit{tấn công từ chối dịch vụ}\footnote{Denial-of-Service attack - DoS attack}, và các vấn đề liên quan đến lạm dụng dịch vụ như \textit{spam} trong một mạng bằng cách yêu cầu một vài công việc bởi phía yêu cầu dịch vụ, thường là một quá trình tiêu tốn tài nguyên (thời gian, bộ nhớ) của máy tính. Thuật ngữ này được sử dụng lần đầu trong một bài báo của \textit{Markus Jakobsson} và \textit{Ari Juels}. Sau đó, nó được phổ biến bởi \textit{Bitcoin}, như là một thuật toán đồng thuận đầu tiên trong mạng phi tập trung không phân quyền.\\

Trong mạng Bitcoin, các nút đào cần thực hiện giải một bài toán "khó" bằng cách tìm ra một con số được gọi là \textit{nonce} sao cho sau khi kết hợp nó với các thông tin đã có của một khối trong chuỗi khối để thực hiện mã hoá băm, ta được một chuỗi băm bắt đầu với một chuỗi các ký tự "0" liên tiếp nhất định, và dãy số này cũng chính là địa chỉ của khối được tạo ra. Nút đầu tiên giải quyết được bài toán trên sẽ quảng bá khối đó lên toàn bộ mạng, các nút khác sẽ xác thực lại tính đúng đắn bằng cách sử dụng hàm băm để mã hoá lại thông tin của khối, so sánh với địa chỉ của khối: Nếu đúng, khối đó sẽ được thêm vào chuỗi, và nút tạo ra khối đó sẽ được nhận phần thưởng. Phần thưởng ở trong mạng Bitcoin chính là một lượng nhỏ đồng tiền mã hoá Bitcoin, và được gửi tới địa chỉ nút nhận ở khối nhất định sau khối vừa được thêm vào.\\

Do sự cải tiến về mặt công nghệ, các máy tính ngày nay có sức mạnh tính toán vô cùng lớn, tạo ra thách thức với các cơ chế đồng thuận dựa trên khối lượng tính toán, trong đó có PoW. Chính vì vậy, độ khó của bài toán mà các nút cần giải quyết ngày càng tăng lên. Tính tới thời điểm tháng 11 năm 2021, số lượng ký tự "0" liên tiếp trong phần đầu của địa chỉ khối trong mạng Bitcoin đã lên tới 7 chữ số, và thời gian để một giao dịch được thực thi trên mạng này xấp xỉ 10 phút. Thời gian thực thi giao dịch lâu khiến cho trải nghiệm của người dùng giảm thấp, đôi khi gây tắc nghẽn hệ thống mạng do nhiều giao dịch không được xử lý. Các cơ chế đồng thuận khác được đưa ra để giải quyết hạn chế này, trong đó có \textit{Bằng chứng cổ phần} - PoS.