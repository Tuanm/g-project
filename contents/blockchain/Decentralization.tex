\subsection{Phi tập trung}

Bằng cách lưu dữ liệu thông qua \textit{mạng ngang hàng}\footnote{Peer-to-peer network (P2P network)}, chuỗi khối giảm được các rủi ro trong lưu trữ dữ liệu tập trung. Chuỗi khối phi tập trung sử dụng \textit{truyền thông điệp tuỳ biến}\footnote{Ad hoc message passing} và \textit{mạng phân tán}\footnote{Distributed networking}. Khi thiếu đi tính phi tập trung, chuỗi khối có thể đối mặt với \textit{tấn công 51\%} (51\% attack) - một thành phần trong mạng có thể kiểm soát nhiều hơn nửa phần còn lại của mạng và điều khiển các bản ghi trong chuỗi khối như ý muốn, trong đó có \textit{chi tiêu gấp đôi} (double-spending).\\

Mạng chuỗi khối ngang hàng giảm thiểu khả năng bị khai thác tại các điểm tập trung dễ bị tấn công nào đó. Các phương pháp bảo mật trong chuỗi khối bao gồm việc sử dụng \textit{mã hoá khoá công khai}\footnote{Public-key cryptography}. Mỗi \textit{khoá công khai} (public key), chuỗi ký tự dài ngẫu nhiên, là một địa chỉ (address) trong chuỗi khối. Các \textit{token} gửi đi khắp mạng sẽ được ghi lại thuộc về một địa chỉ nào đó của chuỗi khối. Mỗi \textit{khoá riêng tư} (private key) giống như một mật khẩu, giúp chủ sở hữu có thể truy cập vào \textit{tài sản số}\footnote{Digial asset} của họ. Dữ liệu được lưu trong chuỗi khối được cho là không thể bị phá vỡ.\\

Mỗi \textit{nút} (node) trong hệ thống phi tập trung giữ một bản sao của chuỗi khối. Chất lượng dữ liệu được duy trì bởi sự \textit{nhân rộng cơ sở dữ liệu}\footnote{Database replication} và sự tin cậy tính toán. Sẽ không có bản sao "chính" nào, cũng sẽ không có người dùng nào "đáng tin" hơn bất cứ ai trong hệ thống. Các giao dịch được gửi đi khắp/quảng bá (broadcasted) trong mạng qua phần mềm. Các thông điệp được gửi dựa trên \textit{nền tảng nỗ lực tốt nhất}\footnote{Best-effort basis}. Các \textit{nút đào}\footnote{Mining node} xác thực giao dịch, thêm chúng vào khối mà nút đó đang tạo, và gửi khối đó đi khắp các nút trong mạng khi đã hoàn thành.
