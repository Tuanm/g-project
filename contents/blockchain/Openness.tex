\subsection{Tính mở}

Các chuỗi khối mở \textit{dễ dùng}\footnote{User-friendly} hơn so với các bản ghi truyền thống - mặc dù mở nhưng vẫn cần truy cập vật lý để xem. Tất cả các chuỗi khối trước đây đều \textit{vô quyền} (permissionless). Nhiều tranh cãi đã nổ ra liên quan đến định nghĩa của chuỗi khối: Liệu một hệ thống riêng tư với xác thực viên được uỷ quyền bởi một \textit{trung tâm uỷ quyền}\footnote{Central authority} có được coi là một chuỗi khối hay không. Phía ủng hộ các chuỗi phân quyền hoặc riêng tư muốn thuật ngữ "chuỗi khối" có thể áp dụng với bất cứ cấu trúc dữ liệu phân dữ liệu thành các \textit{khối đóng dấu thời gian}\footnote{Time-stamped block}. Phía phản đối điều này khẳng định các \textit{hệ thống phân quyền} (permissioned system) giống như cơ sở dữ liệu truyền thống, không hỗ trợ xác thực dữ liệu phi tập trung, không thể chống lại sự giả mạo và sửa đổi.

\subsubsection{Sự vô quyền}
Một lợi thế của một mạng chuỗi khối mở, vô quyền, hoặc công khai là không cần bảo vệ chống lại các tác nhân xấu, không cần kiểm soát truy cập. Nghĩa là, các ứng dụng có thể được thêm vào mạng mà không cần sự chấp thuận và sự tin tưởng của các nút khác trong mạng, sử dụng chuỗi khối như là một \textit{lớp vận chuyển}\footnote{Transport layer}.

\subsubsection{Chuỗi khối phân quyền/riêng tư}
Các chuỗi khối phân quyền sử dụng một lớp điều khiển truy cập để quản lý những ai truy cập vào mạng. Trái với mạng chuỗi khối công cộng, xác thực viên trong mạng chuỗi khối riêng tư được kiểm tra chủ của mạng. Chủ của mạng không dựa vào các \textit{nút ẩn danh}\footnote{Anonymous node} để xác thực giao dịch hay các quyền lợi từ \textit{hiệu ứng mạng}\footnote{Network effect}. Các chuỗi khối phân quyền còn được biết đến với cái tên \textit{chuỗi khối consortium}\footnote{Consortium blockchain}.