\subsection{Giải pháp}

Giải pháp chung phổ biến hiện nay là xây dựng cơ sở dữ liệu chung về văn bằng giữa các cơ sở giáo dục. Nhà tuyển dụng có thể dễ dàng tra cứu thông tin văn bằng khi truy cập vào hệ thống sử dụng cơ sở dữ liệu này. Tuy nhiên, vấn đề hiện hữu là sử dụng cơ sở dữ liệu chung truyền thống tiềm ẩn rất nhiều rủi ro về dữ liệu. Việc nhiều bên truy cập và chỉnh sửa dữ liệu (các cơ sở giáo dục có quyền như nhau đối với cơ sở dữ liệu này) có thể phát sinh mất mát và ảnh hưởng đến dữ liệu của các bên khác. Ngoài ra, nếu giao quyền cập nhật dữ liệu cho một hoặc một số lượng hạn chế các bên tham gia, quy trình rà soát và sửa sai có thể kéo dài và các yêu cầu thay đổi dữ liệu được gửi từ các bên không có quyền cập nhật sẽ không được xử lý kịp thời.\\

Không phải nói quá, chuỗi khối giải quyết quá tốt các bài toán về cơ sở dữ liệu chung, đặc biệt là khi tính minh bạch của dữ liệu được ưu tiên. Đặc biệt, tương tác với chuỗi khối ngày càng trở nên đơn giản với \textit{DApp} - các ứng dụng phi tập trung với giao diện thân thiện, dễ dùng, đồng thời tốc độ truy xuất thông tin chấp nhận được, việc ứng dụng chuỗi khối càng được ưa chuộng. Việc lưu trữ thông tin văn bằng giáo dục trên chuỗi khối đảm bảo được dữ liệu không thể bị chỉnh sửa, mọi người đều dễ dàng truy cập. Nhiệm vụ cấp phát văn bằng được đưa về phía từng cơ sở giáo dục, lưu trữ trên mạng chuỗi khối chung, không thể chỉnh sửa. Đó là ưu điểm khi việc quản lý văn bằng sẽ không quá tập trung vào một số bên nhất định, nhưng cũng là thách thức cho các cơ sở giáo dục trong vấn đề xây dựng hệ thống tích hợp với mạng chuỗi khối một cách hợp lý và tốn không quá nhiều chi phí.