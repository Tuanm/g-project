\newpage
\subsection{Xây dựng hệ thống}

Với yêu cầu của bài toán, ta cần xây dựng hệ thống quản lý văn bằng cho các cơ sở giáo dục, và hệ thống tra cứu thông tin văn bằng cho phía doanh nghiệp (hoặc người có nhu cầu tra cứu).\\

\begin{figure}[ht]
    \centering
    \includegraphics[width=400px]{images/system-overview.png}
    \caption{Sơ đồ tổng quan các hệ thống và mạng chuỗi khối}
\end{figure}

Những hệ thống này sẽ tương tác với các hợp đồng thông minh trên mạng chuỗi khối, bao gồm \textit{Bể văn bằng} (Certs) và \textit{Ví đa chữ ký} (MultiSig. Wallet).

\subsubsection{Bể văn bằng}
\textit{Bể văn bằng} (hay \textit{bể chứng chỉ}) là một hợp đồng thông minh nắm giữ thông tin các văn bằng được lưu trữ trên mạng chuỗi khối. Thông tin được tổ chức theo cấu trúc dạng cây, nắm giữ địa chỉ ví của các cơ sở giáo dục và danh sách các văn bằng cơ sở đó đã cấp phát.\\

Để lưu trữ văn bằng trên mạng chuỗi khối, các cơ sở giáo dục cần sử dụng một địa chỉ ví để tương tác với hợp đồng thông minh được triển khai trên mạng đó. Mỗi cơ sở phát hành văn bằng sẽ có một địa chỉ ví xác định, và trên cấu trúc cây được mô tả như trên, địa chỉ này sẽ được lưu trữ ở các \textit{nút bậc 1} (level-1 node), ký hiệu \texttt{School Address}; thông tin văn bằng được thể hiện qua các số hiệu tương ứng với các \textit{nút lá} (leaf) với ký hiệu \texttt{Certificate No.}.\\

\begin{figure}[!ht]
    \centering
    \includegraphics[width=400px]{images/certs-tree.png}
    \caption{Cấu trúc lưu trữ thông tin trong \textit{bể văn bằng}}
\end{figure}

\textit{Bể văn bằng} cung cấp một số chức năng liên quan đến lưu thông tin, tra cứu thông tin văn bằng, phụ lục văn bằng, thông tin của các cơ sở giáo dục, tổ chức cấp phát văn bằng, chứng chỉ. Tuy nhiên, trong phạm vi bài báo cáo này, ta quan tâm đến hai chức năng cơ bản:
\begin{itemize}
    \item Lưu thông tin văn bằng
    \item Xem thông tin văn bằng
\end{itemize}

Đối với việc \textit{lưu thông tin}, địa chỉ ví của cơ sở (hay người) gửi yêu cầu lưu thông tin văn bằng sẽ được lấy làm \texttt{School Address}, và thông tin được gửi khi tương tác với hợp đồng thông minh sẽ được lưu tương ứng với số hiệu \texttt{Certificate No.}. Như vậy, sẽ không có tình huống một cơ sở giáo dục phát hành văn bằng với địa chỉ của cơ sở khác, trường hợp nhầm lẫn do vô ý hoặc có chủ đích không thể xảy ra.\\

Để \textit{xem thông tin}, người tra cứu cần cung cấp địa chỉ ví của cơ sở phát hành văn bằng (\texttt{School Address}), và số hiệu văn bằng của cơ sở đó (\texttt{Certificate No.}).

\subsubsection{Ví đa chữ ký}
\textit{Ví đa chữ ký} là một hợp đồng thông minh với mục đích tăng tính bảo mật cho quá trình tương tác thay đổi thông tin văn bằng trên chuỗi khối.\\

Với việc "đẩy" thông tin văn bằng lên mạng chuỗi khối một cách thông thường, mỗi cơ sở giáo dục sử dụng địa chỉ ví của một cá nhân đại diện để tương tác, hoặc lựa chọn một địa chỉ ví và sử dụng chung cho cá nhân trong cơ sở. Điều này đảm bảo mỗi cơ sở cấp phát chứng chỉ có một địa chỉ \texttt{School Address} duy nhất. Tuy nhiên, khi nhiều cá nhân cùng dùng một địa chỉ ví, khả năng mất cắp tài sản liên kết với địa chỉ này càng lớn, đặc biệt khi nó còn được sử dụng trong các giao dịch khác có giá trị về tài chính (như địa chỉ sở hữu tiền mã hoá với giá trị cao trên các \textit{sàn giao dịch}\footnote{Exchange}, hay liên kết với các \textit{DApp} khác). Do đó, một cơ chế giúp giảm thiểu khả năng nhiều người cùng sở hữu một địa chỉ ví và có thể sử dụng địa chỉ ví để xác thực thông tin cơ sở cấp phát văn bằng là vô cùng cần thiết. \textit{Ví đa chữ ký} ra đời để giải quyết vấn đề này.\\

Không giống với các \textit{hệ thống xác thực đa chữ ký}\footnote{Multi-signature authentication system} khi ít nhiều phụ thuộc vào các cơ chế xác thực phức tạp, \textit{ví đa chữ ký} sử dụng các tính năng, lợi thế của hợp đồng thông minh và mạng chuỗi khối. Ở \textit{ví đa chữ ký}, mỗi hành động cần thực thi (ở đây là việc cấp phát văn bằng) yêu cầu một số lượng nhất định sự đồng ý từ cá nhân. Địa chỉ ví của các cá nhân này đã được thêm vào danh sách "thành viên" ngay từ khi hợp đồng thông minh này được triển khai, và họ được coi như các "cổ đông" của "doanh nghiệp" cấp phát văn bằng khi có "tiếng nói" trong các "hoạt động" ở đây. Mỗi cơ sở cấp phát văn bằng sử dụng một \textit{ví đa chữ ký} duy nhất, và địa chỉ của hợp đồng thông minh này đại diện cho địa chỉ ví của cả cơ sở đó. Các văn bằng cần được đẩy lên \textit{bể văn bằng} sẽ được một cá nhân trong cơ sở gửi lên "ví" này. Các thành viên khác trong cơ sở có thể xem thông tin các văn bằng được gửi lên, và đưa ra biểu quyết "đồng ý" hay "không đồng ý" trên hợp đồng thông minh. Khi số lượng đồng ý đạt ngưỡng nhất định (được thiết lập từ đầu), các văn bằng đó được đẩy lên "bể", và thông tin được lưu trữ trên mạng chuỗi khối.\\

