\subsection{Giới thiệu bài toán}

Từ lâu, các văn bằng giáo dục là một trong những mục tiêu của quá trình học tập không chỉ ở Việt Nam. Một "tấm bằng" sẽ ghi lại kết quả của một giai đoạn học và tích luỹ kiến thức của cá nhân, thể hiện bằng những điểm số qua các kỳ thi hay các bài kiểm tra cụ thể.\\

Với văn bằng ở các cấp độ giáo dục từ Trung học trở xuống, các thông tin thường được ghi là kết quả đánh giá tổng thể của một cấp học, đi kèm là một số giấy tờ bổ sung chi tiết về điểm số của học sinh và nhận xét của giáo viên, nhà trường đối với học sinh đó (thường được gọi là \textit{Học bạ}). Đối với các văn bằng ở các cấp độ cao hơn như Trung cấp, Cao đẳng, Đại học, điểm số tổng kết và đánh giá năng lực của cơ sở giáo dục đối với học viên/sinh viên sẽ được thể hiện, cùng với đó là đính kèm bảng điểm chi tiết các học phần trong chương trình đào tạo mà học viên theo học tại cơ sở giáo dục đó (được gọi là \textit{Phụ lục văn bằng}).\\

Ngoài văn bằng do các cơ sở giáo dục cấp phát, nhiều cơ quan, tổ chức cũng phát hành văn bằng đánh giá kết quả của cá nhân hoặc một nhóm cá nhân theo một lĩnh vực nào đó, thường dưới dạng các \textit{chứng chỉ}, hay \textit{giấy chứng nhận}.\\

Trong các hoạt động tuyển dụng nhân sự, văn bằng giáo dục (cùng với chứng chỉ) đem đến cho nhà tuyển dụng (các doanh nghiệp, cơ quan, tổ chức) cái nhìn tổng quát đầu tiên về năng lực của ứng viên tương ứng dựa trên đánh giá thể hiện qua thông tin trên văn bằng đó. Những ứng viên với một văn bằng cùng những thông tin tích cực có lợi thế rất lớn trong cuộc đua trở thành "người được chọn", bên cạnh việc thể hiện khả năng của mình trong quá trình làm việc. Việc có được sự đánh giá tốt từ những cơ sở giáo dục chất lượng, có uy tín cũng đem đến sự tin tưởng của nhà tuyển dụng đối với thông tin được ghi trong văn bằng.\\

Tuy nhiên, quy trình xác thực tính đúng đắn của một văn bằng giáo dục tại Việt Nam hiện nay gặp rất nhiều khó khăn. Phần nhiều là vì, các cơ sở giáo dục tại nước ta không công khai thông tin các văn bằng đã cấp phát bởi nhiều lý do. Thêm nữa, sự thiếu hụt các cổng tra cứu công cộng về văn bằng giáo dục cũng khiến các nhà tuyển dụng phải đặt niềm tin "tạm thời" vào một tờ bìa "có vẻ đáng tin cậy". Tính đến thời điểm bài báo cáo này được viết, \textit{Bộ Lao động, Thương binh và Xã hội} đã cung cấp \href{https://vanbang.gdnn.gov.vn/}{\textit{Trang thông tin tra cứu văn bằng Giáo dục nghề nghiệp}}. Tuy nhiên, đó vẫn là chưa đủ. Một hệ thống tra cứu thông tin về văn bằng với độ tin cậy cao và dễ sử dụng, áp dụng được trên mọi cơ sở giáo dục, tổ chức cấp phát văn bằng trở nên hết sức cần thiết.

