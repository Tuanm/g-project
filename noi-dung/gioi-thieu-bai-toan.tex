\chapter{Tổng quan về bài toán cấp phát nội dung số}

Đã từ lâu, các doanh nghiệp cũng như các cá nhân bắt đầu lựa chọn các nền tảng đám mây cho mục đích lưu trữ dữ liệu. Những nhà cung cấp dịch vụ phổ biến hiện nay có thể kể đến như Google với \textit{Google Drive}, Microsoft với \textit{OneDrive}, DropBox, vân vân. Các dịch vụ này tương đối quen thuộc với hầu hết mọi người, hỗ trợ lên tới \textit{15GiB} dữ liệu cho người dùng không trả phí (miễn phí).\\

Do phổ biến như vậy, các doanh nghiệp dần triển khai làm việc trên các dịch vụ này. Dễ dàng nhận thấy rằng, những chứng nhận (như chứng chỉ, văn bằng, huy chương của các cuộc thi, vân vân) đã có thể lưu trữ và phân phối trên nền tảng số. Nhiều doanh nghiệp thậm chí đã chấm dứt cấp phát dưới dạng vật lý (giấy tờ, các vật liệu giả kim hoặc kim loại, vân vân) để giảm thiểu chi phí sản xuất. Cũng có nhiều doanh nghiệp tự xây dựng hệ thống lưu trữ đám mây cho riêng mình để thuận tiện quản lý.\\

Tuy nhiên, việc phụ thuộc vào một nhà cung cấp dịch vụ cũng đem đến rủi ro. Với sự gia tăng ngày càng nhiều của các vụ tấn công nhằm vào các nhà cung cấp dịch vụ đám mây, "khách hàng" chưa hoàn toàn yên tâm trong việc lưu trữ dữ liệu. Không ít các cuộc tấn công như vậy khiến dữ liệu bị sai lệch, không thể khôi phục lại được, hoặc có thể khôi phục lại nhưng dữ liệu không được như ban đầu. Rất nhiều giải pháp khác được thảo luận nhằm đảm bảo cho việc lưu trữ nội dung trên nền tảng số trở lên an toàn hơn.\\