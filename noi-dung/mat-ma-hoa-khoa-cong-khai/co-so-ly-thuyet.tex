\section{Tổng quan về mật mã hoá khoá công khai}

Mật mã hóa khóa công khai là một dạng mật mã hóa cho phép người sử dụng trao đổi các thông tin mật mà không cần phải trao đổi các khóa chung bí mật trước đó. Điều này được thực hiện bằng cách sử dụng một cặp khóa có quan hệ toán học với nhau là khóa công khai và khóa cá nhân (hay khóa bí mật).\\

Thuật ngữ mật mã hóa khóa bất đối xứng thường được dùng đồng nghĩa với mật mã hóa khóa công khai mặc dù hai khái niệm không hoàn toàn tương đương. Có những thuật toán mật mã khóa bất đối xứng không có tính chất khóa công khai và bí mật như đề cập ở trên mà cả hai khóa (cho mã hóa và giải mã) đều cần phải giữ bí mật.\\

Trong mật mã hóa khóa công khai, khóa cá nhân phải được giữ bí mật trong khi khóa công khai được phổ biến công khai. Trong 2 khóa, một dùng để mã hóa và khóa còn lại dùng để giải mã. Điều quan trọng đối với hệ thống là không thể tìm ra khóa bí mật nếu chỉ biết khóa công khai.\\

Hệ thống mật mã hóa khóa công khai có thể sử dụng với các mục đích:
\begin{itemize}
    \item Mã hóa: giữ bí mật thông tin và chỉ có người có khóa bí mật mới giải mã được.
    \item Tạo chữ ký số: cho phép kiểm tra một văn bản có phải đã được tạo với một khóa bí mật nào đó hay không.
    \item Thỏa thuận khóa: cho phép thiết lập khóa dùng để trao đổi thông tin mật giữa 2 bên.
\end{itemize}

Thông thường, các kỹ thuật mật mã hóa khóa công khai đòi hỏi khối lượng tính toán nhiều hơn các kỹ thuật mã hóa khóa đối xứng nhưng những lợi điểm mà chúng mang lại khiến cho chúng được áp dụng trong nhiều ứng dụng.

\subsection{Tính an toàn}

Về khía cạnh an toàn, các thuật toán mật mã hóa khóa bất đối xứng cũng không khác nhiều với các thuật toán mã hóa khóa đối xứng. Có những thuật toán được dùng rộng rãi, có thuật toán chủ yếu trên lý thuyết; có thuật toán vẫn được xem là an toàn, có thuật toán đã bị phá vỡ... Cũng cần lưu ý là những thuật toán được dùng rộng rãi không phải lúc nào cũng đảm bảo an toàn. Một số thuật toán có những chứng minh về độ an toàn với những tiêu chuẩn khác nhau. Nhiều chứng minh gắn việc phá vỡ thuật toán với những bài toán nổi tiếng vẫn được cho là không có lời giải trong thời gian đa thức. Nhìn chung, chưa có thuật toán nào được chứng minh là an toàn tuyệt đối (như hệ thống mật mã sử dụng một lần). Vì vậy, cũng giống như tất cả các thuật toán mật mã nói chung, các thuật toán mã hóa khóa công khai cần phải được sử dụng một cách thận trọng.

\subsection{Các ứng dụng}

Ứng dụng rõ ràng nhất của mật mã hóa khóa công khai là bảo mật: một văn bản được mã hóa bằng khóa công khai của một người sử dụng thì chỉ có thể giải mã với khóa bí mật của người đó.\\

Các thuật toán tạo chữ ký số khóa công khai có thể dùng để nhận thực. Một người sử dụng có thể mã hóa văn bản với khóa bí mật của mình. Nếu một người khác có thể giải mã với khóa công khai của người gửi thì có thể tin rằng văn bản thực sự xuất phát từ người gắn với khóa công khai đó.\\

Các đặc điểm trên còn có ích cho nhiều ứng dụng khác như: tiền điện tử, thỏa thuận khóa...

