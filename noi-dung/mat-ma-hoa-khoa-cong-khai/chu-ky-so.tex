\section{Chữ ký số}

Ứng dụng điển hình của mật mã khóa công khai là xác thực dữ liệu thông qua việc sử dụng chữ ký số. Chữ ký số thực chất là một chuỗi nhị phân đặc trưng cung cấp tính toàn vẹn, bằng chứng về nguồn gốc, danh tính và trạng thái của một tài liệu, giao dịch hoặc thông điệp điện tử. Chúng được sử dụng rộng rãi trong nhiều giao thức cho mục đích xác thực và đã được chứng minh là rất hữu ích và an toàn.\\

Chữ ký số là không thể sửa chữa và dễ dàng kiểm chứng nhờ sử dụng mật mã khóa công khai. Ngoài ra, do khóa riêng trong hệ mật khóa công khai chỉ liên kết đến một người dùng duy nhất, và chỉ có người dùng đó giữ khóa riêng, vì vậy chữ ký số cũng đảm bảo khả năng chống chối bỏ, có nghĩa là khi đã ký số lên dữ liệu thì người dùng không thể chối bỏ việc đã ký lên nó, và do vậy, chúng có thể ràng buộc về mặt pháp lý như chữ ký thông thường.\\

Bình thường, khi chúng ta sử dụng hệ mật mã khóa công khai để mã hóa và truyền dữ liệu, chúng ta sử dụng khóa công khai của người nhận để mã hóa, và do đó người nhận có thể dùng khóa riêng tương ứng của mình để giải mã dữ liệu, và chỉ duy nhất người nhận có khả năng làm việc đó. Tuy nhiên, trong chữ ký số, mọi chuyện lại ngược lại.\\

Chúng ta muốn xác nhận rằng thông điệp này đúng là do bản thân chúng ta ký vào nó, và mọi người đều có thể xác thực được. Do vậy, chữ ký số cần được giải mã bởi mọi người. Để đáp ứng điều này, chúng ta cần cho họ khóa chung (hay khóa công khai) và việc mã hóa diễn ra bằng khóa riêng bí mật của chính chúng ta.\\

Khi bạn mã hóa một cái gì đó bằng khóa riêng của mình, bất kỳ ai cũng có thể giải mã nó bằng khóa công khai, điều này nghe thật vô dụng, nhưng điều này đóng vai trò là bằng chứng cho việc bạn đã mã hóa dữ liệu. Nếu người khác không thể giải mã đúng dữ liệu bằng khóa công khai của bạn, thì hoặc là dữ liệu đã bị thay đổi và mất tính toàn vẹn hoặc là bạn không hề ký và tạo ra dữ liệu đó. Đây được gọi là chữ ký số.\\

Trên thực tế, việc tạo và xác nhận chữ ký số phức tạp hơn rất nhiều, nó đòi hỏi có các thuật toán tạo chữ ký, sinh khóa, biến đổi dữ liệu, và thuật toán xác minh chữ ký. Mọi thứ tuân theo các lược đồ chữ ký số. Tuy nhiên về cơ bản nguyên lý hoạt động của chữ ký số đều tuân theo nguyên tắc trên.