\section{Hệ mật RSA}

Thuật toán được Ron Rivest, Adi Shamir và Len Adleman mô tả lần đầu tiên vào năm 1977 tại Học viện Công nghệ Massachusetts (MIT). Tên của thuật toán lấy từ 3 chữ cái đầu của tên 3 tác giả. Đây là thuật toán đầu tiên phù hợp với việc tạo ra chữ ký điện tử đồng thời với việc mã hóa. Nó đánh dấu một sự tiến bộ vượt bậc của lĩnh vực mật mã học trong việc sử dụng khóa công cộng. RSA đang được sử dụng phổ biến trong thương mại điện tử và được cho là đảm bảo an toàn với điều kiện độ dài khóa đủ lớn.\\

Thuật toán RSA được MIT đăng ký bằng sáng chế tại Hoa Kỳ vào năm 1983 (Số đăng ký 4,405,829). Bằng sáng chế này hết hạn vào ngày 21 tháng 9 năm 2000. Tuy nhiên, do thuật toán đã được công bố trước khi có đăng ký bảo hộ nên sự bảo hộ hầu như không có giá trị bên ngoài Hoa Kỳ. Ngoài ra, nếu như công trình của Clifford Cocks đã được công bố trước đó thì bằng sáng chế RSA đã không thể được đăng ký.\\

Thuật toán dựa trên độ khó của bài toán phân tích một số thành nhân tử.

\subsection{Quá trình tạo khoá cho hệ mật RSA}

Giả sử hai người, An và Bình, cần trao đổi thông tin bí mật thông qua một kênh không an toàn (ví dụ như Internet). Với thuật toán RSA, An đầu tiên cần tạo ra cho mình cặp khóa gồm khóa công khai và khóa bí mật theo các bước sau:
\begin{enumerate}
    \item Chọn hai số nguyên tố lớn khác nhau $p$ và $q$.
    \item Tính tích $n=pq$.
    \item Tính một số giả nguyên tố của $n$ bằng phi hàm Carmichael như sau:
    $$\lambda(n)=\mathrm{GCD}(\lambda(p),\lambda(q))=\mathrm{GCD}(p-1,q-1).$$
    Giá trị này sẽ được giữ bí mật.
    \item Chọn một số tự nhiên $e$ trong khoảng $(1,\lambda(n))$ sao cho $\mathrm{LCM(e,\lambda(n))}=1$, tức là $e$ và $\lambda(n)$ nguyên tố cùng nhau.
    \item Tính toán số $d$ sao cho $de\equiv1\pmod{\lambda(n)}$. Số $d$ được gọi là số nghịch đảo modulo của $e$ (theo modulo $\lambda(n)$).
\end{enumerate}

Khoá công khai sẽ là bộ số $(n,e)$, và khoá bí mật sẽ là bộ số $(n,d)$. Chúng ta cần giữ khoá bí mật thật cẩn thận, cũng như các số nguyên tố $p$ và $q$ vì từ đó có thể tính ra các khoá khá dễ dàng.

\subsection{Mã hoá và giải mã}

\subsubsection*{Mã hoá}

Giả sử Bình muốn gửi đoạn thông tin $M$ cho An. Đầu tiên Bình chuyển $M$ thành một số $m<n$ theo một hàm có thể đảo ngược (từ $m$ có thể xác định lại $M$) được thỏa thuận trước. Quá trình này được mô tả ở phần \textbf{Chuyển đổi văn bản rõ}.\\

Lúc này Bình có $m$ và biết $n$ cũng như $e$ do An gửi. Bình sẽ tính $c$ là bản mã hóa của $m$ theo công thức:
$$
c\equiv m^e\pmod{n}.
$$

Hàm trên có thể tính dễ dàng sử dụng phương pháp tính hàm mũ (theo modulo) bằng (thuật toán bình phương và nhân).\\

Cuối cùng Bình gửi $c$ cho An giải mã.

\subsubsection*{Giải mã}

An nhận $c$ từ Bình và biết khóa bí mật $d$. An có thể tìm được $m$ từ $c$ theo công thức sau:
$$
c^d\equiv(m^e)^d\pmod{n}\equiv m^{de}\pmod{n}.
$$

Do $de\equiv 1\pmod{\lambda(n)}$, hay $de\equiv1\pmod{p-1}$ và $de\equiv1\pmod{q-1}$ (theo Định lý Fermat nhỏ), nên:
$$
m^{de}\equiv m\pmod{p},\;m^{de}\equiv m\pmod{q}.
$$

Lại do $p$ và $q$ nguyên tố cùng nhau, áp dụng Định lý Phần dư Trung Hoa, ta có:
$$
m^{de}\equiv m\pmod{pq},
$$
hay:
$$
c^d\equiv m\pmod{n}.
$$

\subsubsection*{Chuyển đổi văn bản rõ}

Trước khi thực hiện mã hóa, ta phải thực hiện việc chuyển đổi văn bản rõ (chuyển đổi từ $M$ sang $m$) sao cho không có giá trị nào của $M$ tạo ra văn bản mã không an toàn. Nếu không có quá trình này, RSA sẽ gặp phải một số vấn đề sau:
\begin{itemize}
    \item Nếu $m=0$ hoặc $m=1$ sẽ tạo ra các bản mã có giá trị là $0$ và $1$ tương ứng.
    \item Khi mã hóa với số mũ nhỏ (chẳng hạn $e=3$) và $m$ cũng có giá trị nhỏ, giá trị $m^e$ cũng nhận giá trị nhỏ (so với $n$). Như vậy phép modulo không có tác dụng và có thể dễ dàng tìm được $m$ bằng cách khai căn bậc $e$ của $c$ (bỏ qua modulo).
    \item RSA là phương pháp mã hóa xác định (không có thành phần ngẫu nhiên) nên kẻ tấn công có thể thực hiện tấn công lựa chọn bản rõ bằng cách tạo ra một bảng tra giữa bản rõ và bản mã. Khi gặp một bản mã, kẻ tấn công sử dụng bảng tra để tìm ra bản rõ tương ứng.
\end{itemize}