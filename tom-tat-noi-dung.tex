\section*{
    \centering
    Tóm tắt nội dung của Đồ án
    \vspace{1.25cm}
}

\subsection*{Mục đích của Đồ án}

Đồ án \textit{Ứng dụng Chuỗi khối trong quản lý văn bằng giáo dục} cung cấp những cái nhìn tổng quan về vấn đề xác thực "bằng thật - bằng giả" giữa nhà tuyển dụng và ứng viên; thực trạng và tính minh bạch của các hệ thống tra cứu hiện có; đồng thời đưa ra giải pháp sử dụng công nghệ \textit{chuỗi khối} cho quá trình cấp phát và tra cứu văn bằng.

\subsection*{Các công nghệ đã sử dụng}

Ngoài cơ sở lý thuyết về \textit{chuỗi khối} cùng với \textit{mật mã hoá bất đối xứng}, phần triển khai của đồ án có sử dụng các công nghệ sau:
\begin{itemize}
    \item \textit{Mạng Ethereum}: Nơi triển khai các \textit{hợp đồng thông minh} - chương trình thực thi trên mạng phân tán, nơi dữ liệu được thay đổi dựa trên tính đồng thuận của toàn mạng.
    \item \textit{Giao thức IPFS}: Một giao thức cho phép lưu trữ và phân phối tập tin trên một mạng ngang hàng phân tán một cách bảo mật và tiết kiệm bộ nhớ.
\end{itemize}

Bên cạnh đó, \textit{hệ thống cấp phát văn bằng} được xây dựng để dễ dàng thao tác với \textit{ví MetaMask} - một chương trình lưu trữ \textit{khoá bí mật} cung cấp các tính năng để tương tác với các \textit{mạng chuỗi khối} một cách thân thiện và an toàn.

